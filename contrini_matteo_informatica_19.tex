% !TeX document-id = {e6d65a87-ee37-4f3f-97f4-f0ea465e516e}
% !TeX TXS-program:compile = txs:///pdflatex/[--shell-escape]

% formato FRONTE RETRO
\documentclass[epsfig,a4paper,11pt,titlepage,twoside,openany]{book}
\usepackage{epsfig}
\usepackage{plain}
\usepackage{setspace}
\usepackage[paperheight=29.7cm,paperwidth=21cm,outer=1.5cm,inner=2.5cm,top=2cm,bottom=2cm]{geometry} % per definizione layout
\usepackage{titlesec} % per formato custom dei titoli dei capitoli

% accenti
\usepackage[utf8]{inputenc}

% conversione virgolette
\usepackage[autostyle, italian=quotes]{csquotes}
\MakeOuterQuote{"}

\singlespacing

\usepackage[italian]{babel}

% immagini a lato
\usepackage{wrapfig}

% Codice
\usepackage{minted}

% per colore LightCyan
\usepackage[svgnames,dvipsnames]{xcolor}

% disegnini
\usepackage{tikz}
% per usare right=of
\usetikzlibrary{positioning}
% frecce
\usetikzlibrary{arrows.meta}

% per rimuovere due punti in caption vuote
% https://tex.stackexchange.com/a/17726/193944
\usepackage[justification=centering]{caption}

% figure affiancate
\usepackage{subcaption}

% bibliografia con url
\usepackage{url}

% \hl
\usepackage{soulutf8}

%Nota. Si ricorda che il numero massimo di facciate e' 30. Nel conteggio delle facciate sono incluse 1) indice 2) sommario 3) capitoli. Dal conteggio delle facciate sono escluse 1) frontespizio 2) ringraziamenti 3) allegati 

\showthe\textwidth

\begin{document}

  % nessuna numerazione
  \pagenumbering{gobble} 
  \pagestyle{plain}

\thispagestyle{empty}

\begin{center}
  \begin{figure}[h!]
    \centerline{\psfig{file=res/logo_unitn_black_center.eps,width=0.6\textwidth}}
  \end{figure}

  \vspace{2 cm} 

  \LARGE{Dipartimento di Ingegneria e Scienza dell’Informazione\\}

  \vspace{1 cm} 
  \Large{Corso di Laurea in\\
    Informatica
  }

  \vspace{2 cm} 
  \Large\textsc{Elaborato finale\\} 
  \vspace{1 cm} 
  \Huge\textsc{Realizzazione di un sistema IoT basato su Android per l'acquisizione multimediale di lezioni universitarie\\}
%  \Large{\it{Sottotitolo (alcune volte lungo - opzionale)}}


  \vspace{2 cm} 
  \begin{tabular*}{\textwidth}{ c @{\extracolsep{\fill}} c }
  \Large{Supervisore} & \Large{Laureando}\\
  \Large{Marco Ronchetti}& \Large{Matteo Contrini}\\
  \end{tabular*}

  \vspace{2 cm} 

  \Large{Anno accademico 2018/2019}
  
\end{center}



  \clearpage

  \thispagestyle{empty}

\begin{center}
  {\bf \Huge Ringraziamenti}
\end{center}

\vspace{4cm}


\emph{
La presente tesi è stata redatta nell'ambito di un tirocinio formativo svolto presso l’azienda AXIA Studio S.r.l.
}

  \clearpage
  \pagestyle{plain} % nessuna intestazione e pie pagina con numero al centro


  % inizio numerazione pagine in numeri arabi
  \mainmatter

    % indice
    \tableofcontents
    \clearpage
    
    
    % gruppo per definizone di successione capitoli senza interruzione di pagina
    \begingroup
      % nessuna interruzione di pagina tra capitoli
      % ridefinizione dei comandi di clear page
      \renewcommand{\cleardoublepage}{} 
      \renewcommand{\clearpage}{} 
      % redefinizione del formato del titolo del capitolo
      % da formato
      %   Capitolo X
      %   Titolo capitolo
      % a formato
      %   X   Titolo capitolo
      
      \titleformat{\chapter}
        {\normalfont\Huge\bfseries}{\thechapter}{1em}{}
        
      \titlespacing*{\chapter}{0pt}{0.59in}{0.02in}
      \titlespacing*{\section}{0pt}{0.20in}{0.02in}
      \titlespacing*{\subsection}{0pt}{0.10in}{0.02in}
      
      % sommario
      \chapter*{Sommario} % senza numerazione
\label{sommario}

\addcontentsline{toc}{chapter}{Sommario} % da aggiungere comunque all'indice

L'oggetto di questa tesi è lo sviluppo di un sistema di cattura e registrazione multimediale di lezioni universitarie, conferenze e seminari. Il sistema prodotto si basa su un dispositivo fisico dotato di apposito hardware per la cattura video, sul quale viene eseguito un apposito software su sistema operativo Android.

Il sistema è stato realizzato nell'ambito del progetto LODE (Lectures On DEmand) del professor Marco Ronchetti (Università di Trento), ed è stato elaborato durante un tirocinio formativo presso l'azienda AXIA STUDIO S.R.L., con l'obiettivo finale di affinare la soluzione e di immetterla sul mercato delle soluzioni IoT accademiche.

Il progetto LODE, nel cui contesto questo lavoro si inserisce, è un progetto sperimentale realizzato in diverse iterazioni presso l'Università di Trento, con l'obiettivo di fornire una soluzione low cost per l'acquisizione video delle lezioni universitarie. Le ultime versioni del sistema prevedono la possibilità di registrare flussi multimediali multipli, in particolare il video in uscita dal computer del docente, il video acquisito da una videocamera che riprende il docente e/o la lavagna, e l'audio di un microfono indossato dal docente.

Uno degli obiettivi principali del sistema è di essere sufficientemente semplice da poter essere comandato direttamente da un professore. Al termine della lezione, un processo di post-elaborazione automatizzabile si occupa di unire i flussi video e audio e di sincronizzarli, per ottenere i contenuti finali da pubblicare in una pagina web accessibile dagli studenti iscritti al corso.

Una funzionalità del sistema prevede anche che lo studente possa catturare durante le lezioni degli screenshot di ciò che viene proiettato in quel momento, con la possibilità di scrivere annotazioni testuali o di disegnare sulle catture.

Nella sua ultima iterazione, il sistema LODE prevede l'uso di un dispositivo fisico, soprannominato LodeBox, che incorpora un single-board computer (computer su una scheda) Raspberry Pi. Tramite apposite estensioni hardware, in particolare un modulo "HDMI to CSI-2", il dispositivo è in grado di acquisire un input HDMI come se si trattasse del video di una videocamera, e di sfruttare funzionalità come l'anteprima su schermo e la registrazione H.264 tramite encoder hardware. Il dispositivo prevede anche la possibilità di collegare un proiettore tramite uscita HDMI, un microfono tramite un adattatore USB, e una videocamera RTSP (Real Time Streaming Protocol) tramite la rete locale.

Questa soluzione si scontra però con delle difficoltà tecniche, che rendono difficile la realizzazione di un sistema affidabile e pronto per il pubblico. Tra le problematiche si evidenziano: la mancanza di storage integrato e duraturo ad alte prestazioni; la scarsa qualità del Wi-Fi; la difficoltà nel gestire situazioni "plug and play" come la disconnessione e la riconnessione del cavo HDMI, che richiederebbero di riavviare il dispositivo forzatamente; la difficoltà nel cambiare il branding software del dispositivo in modo che non sia evidente la presenza del sistema operativo Linux Raspbian.

%La verifica degli aspetti appena citati è stata la prima attività da me svolta, e ha portato ad alcuni miglioramenti, tra cui la notevole riduzione dello spazio di archiviazione richiesto per i file video.

Ci si è spostati quindi sulla ricerca di SBC (Single-Board Computer) alternative e più adatte per la realizzazione di applicazioni multimediali. La maggior parte delle SBC con funzioni multimediali è basata su sistema operativo Android, e in alcuni casi sono forniti SDK dedicati per sfruttare funzioni specifiche del dispositivo.

Gran parte del mio lavoro si è quindi concentrato sullo sviluppo di alcune applicazioni per Android, per verificare la fattibilità di una versione di LodeBox su hardware Android, utilizzando sia Kotlin che Java come linguaggi di programmazione.

La peculiarità fondamentale di molte board Android con input HDMI è che permettono di sfruttare direttamente le API di Android per l'acquisizione di video e immagini, consentendo quindi di scrivere codice che non sia strettamente legato all'hardware. Il sistema operativo Android prevede due versioni delle API per la "fotocamera", in particolare la classe \texttt{android.hardware.Camera} e le classi contenute in \texttt{android.hardware.camera2}. A partire da Android 5.0 (livello API 21), la classe \texttt{Camera} è deprecata ed è invece consigliato l'uso delle API \texttt{camera2}. Le differenze tra le due versioni sono notevoli: la seconda è più avanzata, ma anche più difficile da usare e non è garantito che ogni dispositivo con Android 5 o superiore la supporti a pieno. Esiste infatti una modalità \texttt{LEGACY} che dà la possibilità di usare le API \texttt{camera2} nonostante l'assenza di un supporto diretto a basso livello per le API.

È stato quindi necessario approfondire il funzionamento di entrambe le versioni delle API, per capire sia quale conviene usare in base all'hardware disponibile, sia quale è tecnicamente possibile usare per poter ottenere tutte le funzionalità richieste dal sistema LODE (ossia la proiezione del video in input, la registrazione in formato compresso e la cattura e invio di screenshot).

Oltre all'input HDMI, c'era la necessità di registrare anche il video di una videocamera esterna. Questa era raggiungibile tramite la rete locale e permetteva di acquisire il video da remoto tramite il protocollo di streaming RTSP. Di conseguenza, il flusso può essere registrato direttamente sulla board, utilizzando una versione della libreria \texttt{ffmpeg} compilata per funzionare su Android.

Un altro punto importante dello sviluppo di un'applicazione embedded per Android è la realizzazione di una modalità \textit{kiosk}, cioè una situazione in cui l'utilizzatore del sistema può utilizzare solo ed esclusivamente una singola applicazione, senza poter accedere alle altre parti del sistema operativo o ad altre funzioni. Questo ha portato a sperimentare diverse soluzioni per assicurarsi che l'applicazione resti sempre a schermo intero, tra cui la modalità "lock task" di Android e l'avvio automatico come applicazione launcher.

Il risultato di questa fase è stato un insieme di applicazioni-prototipo che sono state sperimentate sul campo durante alcune lezioni presso l'Università di Trento, permettendo di raccogliere importanti riscontri sul funzionamento del sistema.

A questo punto la fattibilità delle funzionalità di base che il sistema deve fornire sono state verificate, e i passi successivi hanno riguardato altri aspetti che possono migliorare il funzionamento del software.

Uno di questi è la possibilità di mettere in pausa e riprendere la registrazione. L'idea considerata è stata quella di non sospendere la registrazione ma di escludere in post-produzione i segmenti di video corrispondenti alle pause. Questo è agevolmente ottenibile grazie a \texttt{ffmpeg}, con l'aiuto di uno script che generi il comando (potenzialmente lungo) per il ritaglio dei segmenti.

Un'altra questione approfondita riguarda la sincronizzazione dei flussi, con lo scopo di evitare che l'audio risulti troppo sfasato rispetto al video, ed evitando quindi che sia visibile lo sfasamento tra voce registrata e il labiale del relatore. Il problema sorge principalmente per la presenza del video RTSP, la cui latenza è difficilmente stimabile\footnote{La latenza è influenzata dalla rete, dai tempi di codifica e dai buffer di trasmissione e ricezione. Non è quindi facile trovare un intervallo di tempo abbastanza preciso per sincronizzare il video con gli altri flussi.}. La soluzione sperimentale proposta prevede di incorporare l'audio nella registrazione video HDMI (in modo da ottenere una naturale sincronizzazione tra i due flussi), e di occuparsi invece di sincronizzare video HDMI e RTSP. L'implementazione sperimentale è stata ottenuta mostrando su schermo un marcatore visivo, che permetta di individuare con precisione un instante comune tra i due flussi video, e di conseguenza sincronizzarli.

Come accennato, una funzione del sistema LODE prevede che lo studente possa prendere appunti in tempo reale durante lo svolgimento delle lezioni, anche catturando degli screenshot di quanto proiettato in quel momento. Per evitare di inviare inutilmente screenshot al server nel caso in cui non ci sia stato nessun cambiamento percepibile nell'immagine, un sistema intelligente potrebbe rilevare le differenze tra i fotogrammi per determinare se l'immagine è nuova o invariata. Dopo aver appurato empiricamente che il confronto di tutti i pixel di due fotogrammi potrebbe risultare computazionalmente dispendioso, un'alternativa è quella di scegliere con un fattore di casualità un numero limitato di pixel "salienti". A livello intuitivo, il metodo sviluppato realizza una griglia con un numero oscillante di righe e colonne, da cui viene estratto un numero limitato di pixel utilizzati per rilevare rapidamente le differenze tra i fotogrammi.

I capitoli di questa tesi approfondiscono più dettagliatamente gli aspetti, le problematiche e le soluzioni proposte che sono state sintetizzate in questo sommario.

%\begin{itemize}
%  \item scelta delle API di Android Camera o Camera2, per l'accesso all'input HDMI del dispositivo;
%  \item implementazione della registrazione di un flusso video RTSP, per poter acquisire una videocamera connessa in rete;
%  \item implementazione della registrazione del video HDMI in ingresso;
%  \item aggiunta della cattura di fotogrammi durante la registrazione della lezione;
%  \item	integrazione dell'applicazione con il web server che permette agli studenti di catturare screenshot;
%  \item	rilevamento della disconnessione e riconnessione dell'input HDMI, e del cambio risoluzione;
%  \item	realizzazione di una “kiosk mode”, in modo da impedire all'utilizzatore di uscire dall'applicazione o di compromettere il sistema o il dispositivo;
%  \item	realizzazione di uno script per ritagliare segmenti dei video registrati, per ottenere la funzionalità di pausa/ripresa della registrazione;
%  \item	sincronizzazione dei flussi video HDMI e RTSP, in particolare mediante l'uso di un marcatore visivo presente in entrambi i flussi;
%  \item	rilevamento ed estrazione delle “slide” dal video HDMI in post-produzione;
%  \item	rilevamento dei cambiamenti al flusso video HDMI, in modo da evitare di inviare al server screenshot che sono già stati inviati;
%  \item	registrazione di audio LPCM e compressione AAC.
%\end{itemize}


     
      % lista dei capitoli
      \needspace{5\baselineskip}
\chapter{Introduzione}
\label{cha:intro}

Questo capitolo introduce il progetto LODE sviluppato presso l'Università di Trento, nel cui contesto questa tesi si inserisce. Sono approfonditi motivazioni e vantaggi, ed è introdotta l'ultima versione del sistema, cioè un dispositivo hardware dal nome LodeBox. Sono infine presentate le principali problematiche legate a LodeBox e l'hardware alternativo considerato per l'evoluzione del progetto.

\section{LODE: introduzione e motivazioni}
\label{sec:intro_lode}

LODE (Lectures On DEmand) è un progetto realizzato dal professor Marco Ronchetti e collaboratori presso l'Università di Trento. Si presenta come una soluzione per l'acquisizione in formato video e audio di lezioni universitarie, con la particolarità di essere una soluzione a basso costo e facilmente manovrabile \cite{ronchetti}.

Le lezioni registrate possono poi essere consultate tramite una pagina web appositamente generata, che combina il video del computer del docente, il video ripreso da una videocamera posizionata in aula e l'audio catturato da un microfono.

I vantaggi che questo sistema offre sono molteplici, tra cui: la possibilità per gli studenti-lavoratori di seguire le lezioni in remoto a qualsiasi orario; la possibilità di recuperare le lezioni in caso di assenze non volontarie (es. malattia); supporto per gli studenti che non comprendono bene la lingua del corso; la possibilità di rivedere porzioni specifiche di qualsiasi lezione in qualsiasi momento, e di verificare quindi la qualità dei propri appunti e il livello di comprensione.

Le versioni più recenti di LODE prevedono anche un'interfaccia web utilizzabile dagli studenti durante lo svolgimento della lezione. Questo strumento permette di catturare degli screenshot di quanto proiettato dal docente in quell'istante, e di scrivere o disegnare annotazioni sulle catture. In questo modo gli studenti sono coinvolti in modo meno passivo e seguono la lezione con più attenzione.

\section{La soluzione LodeBox}
\label{sec:intro_lodebox}

\begin{wrapfigure}{R}{0.3\textwidth}
	\vspace{-12pt}
	\includegraphics[width=0.3\textwidth]{res/lodebox}
	\caption{\label{fig:lodebox} La scatola di LodeBox.}
\end{wrapfigure}

L'ultima versione di LODE prevede l'utilizzo di una board Raspberry Pi per eseguire il software di acquisizione. Il dispositivo è inserito in una piccola scatola che espone dei connettori verso l'esterno. Tra questi sono presenti un ingresso HDMI per collegare un computer e un'uscita HDMI per collegare un proiettore o uno schermo. Le porte USB permettono di collegare un telecomando e un "dongle" per l'acquisizione dell'audio di un microfono con jack 3,5 mm.

La scatola include un modulo "HDMI to MIPI CSI-2", che permette di convertire il segnale HDMI nel formato seriale della fotocamera. Il video HDMI è infatti acquisito utilizzando la libreria \texttt{PiCamera} per Python, che permette di acquisire il segnale come se si trattasse del video di una fotocamera. La libreria semplifica di molto lo sviluppo, perché espone delle funzioni di alto livello per svolgere operazioni come abilitare l'anteprima del video a schermo intero, la registrazione con un encoder H.264\footnotemark{} con accelerazione hardware e la cattura di screenshot.

\footnotetext{H.264, conosciuto anche come AVC o MPEG-4 Part-10, è il più popolare codec di compressione video. Nato nel 2003, è ancora oggi lo standard de facto in numerosi ambiti, tra cui lo streaming video.}

Sempre in ottica di riduzione dei costi, LodeBox prevede la ripresa del docente tramite una qualsiasi videocamera IP (wireless o cablata) che supporti il protocollo RTSP (Real Time Streaming Protocol). La registrazione avviene sul dispositivo utilizzando \texttt{ffmpeg}/\texttt{avconv}\footnotemark{} in modalità copia (senza ricodifica), richiedendo quindi un uso molto basso di risorse.

\footnotetext{\emph{ffmpeg} è uno strumento estremamente popolare per l'elaborazione di video e audio tramite linea di comando. Supporta numerosi formati, codec e protocolli, e permette di effettuare operazioni quali il muxing, transmuxing, transcoding e altre funzioni più avanzate. \emph{avconv} è un fork di ffmpeg nato per via di divergenze sui metodi di sviluppo, ed è stato incluso per un breve periodo in alcune distribuzioni Linux in sostituzione di ffmpeg.}

Raspberry Pi non prevede spazio di archiviazione interno, e richiede quindi di utilizzare una scheda microSD o una chiavetta USB per memorizzare le registrazioni. Le memorie di tipo flash hanno però vita limitata, a seconda di quanto intensivamente sono utilizzate, e questo riduce di conseguenza l'affidabilità a lungo termine del sistema. Inoltre, la soluzione di usare la porta USB non risulta percorribile, per via delle prestazioni molto scarse che impediscono di registrare tre flussi in contemporanea.

Un altro svantaggio di questa soluzione è il livello di "fault tolerance" in caso di eventi come la disconnessione (volontaria o meno) del cavo HDMI in ingresso. Dato che l'input HDMI è mappato sull'interfaccia della fotocamera, non è previsto che la fotocamera possa essere improvvisamente scollegata. Secondo le prove effettuate, risulta molto difficile rilevare in modo affidabile la disconnessione del cavo HDMI. Nei casi in cui è possibile, non risulta invece fattibile il recupero dell'applicazione, perché le operazioni sulla \texttt{PiCamera} sollevano eccezioni o bloccano indefinitamente l'esecuzione del codice.

Infine, un altro problema è posto dalla presenza del modulo di conversione HDMI-CSI, la cui disponibilità e compatibilità a lungo termine non sono garantite.

\section{Hardware alternativo}
\label{sec:intro_hardware}

Per poter evolvere LodeBox in una soluzione più affidabile e adatta alla produzione e distribuzione, si sono quindi cercate SBC (Single-Board Computer) alternative, preferibilmente pensate per la realizzazione di applicazioni multimediali.

Dal punto di vista tecnico, i principali vincoli per la scelta di una nuova scheda erano la presenza dell'input HDMI e di un encoder H.264 hardware, la possibilità di collegare un microfono e il costo accessibile.

Le soluzioni considerate erano inoltre di tipo pre-industriale, quindi non necessariamente già presenti sul mercato ma personalizzabili per soddisfare i requisiti del prodotto, e offrivano anche determinate garanzie di disponibilità e supporto per un periodo di tempo prolungato.

Molte board con sistema operativo Android soddisfano queste caratteristiche, e consentono in aggiunta di avere a disposizione una piattaforma nota, documentata, di facile sviluppo e per cui sono disponibili molte risorse. In molti casi l'input HDMI è reso disponibile tramite l'interfaccia della fotocamera CSI, da cui deriva la possibilità di acquisire l'input tramite le API di Android per l'accesso alla fotocamera.


      \chapter{Acquisizione video HDMI}
\label{cha:hdmi}
Lorem ipsum dolor sit amet.

%Esiste infatti una modalità \texttt{LEGACY} che dà la possibilità di usare le API \texttt{camera2} nonostante l'assenza di un supporto diretto a basso livello per le API.


\section{Le API \texttt{android.hardware.Camera}}
\label{sec:hdmi_camera}
Lorem ipsum dolor sit amet.


\section{Le API \texttt{android.hardware.camera2.*}}
\label{sec:hdmi_camera2}
Lorem ipsum dolor sit amet.\cite{camera2}



      \chapter{Acquisizione video RTSP}
\label{cha:rtsp}
Lorem ipsum dolor sit amet.

\section{Il protocollo RTSP}
\label{sec:rtsp_protocollo}

Lorem ipsum dolor sit amet.

\section{La libreria MobileFFmpeg}
\label{sec:rtsp_ffmpeg}

Lorem ipsum dolor sit amet.


      \chapter{Realizzazione della modalità "kiosk"}
\label{cha:kiosk}

Nell'ambito dei sistemi \emph{embedded} si utilizza spesso la locuzione \emph{modalità kiosk} per indicare tutte le situazioni in cui il sistema deve comportarsi come un "chiosco digitale" e limitare l'utilizzo a specifiche funzioni. Su Android si possono adottare diverse tecniche per ottenere una modalità simile, nascondendo quindi la presenza del sistema operativo Android. In questo capitolo sono presentate le tecniche sperimentate, tra cui la modalità a schermo intero, la modifica dell'\emph{overscan} di sistema, la modalità \emph{lock task}, l'avvio automatico dell'applicazione e la modifica dell'animazione di avvio del sistema.

\section{La modalità a schermo intero}
\label{sec:kiosk_fullscreen}

A partire da Android 4.1, è stata introdotta una gestione granulare della modalità a schermo intero. Si tratta di base della possibilità di nascondere, utilizzando appositi \emph{flag}, la barra di stato e la barra di navigazione di Android, presenti rispettivamente nella parte superiore e inferiore dello schermo.

\begin{minted}{java}
private void goFullScreen() {
    getWindow().getDecorView().setSystemUiVisibility(
        View.SYSTEM_UI_FLAG_HIDE_NAVIGATION
        | View.SYSTEM_UI_FLAG_FULLSCREEN);
}
\end{minted}

Questo metodo va chiamato sia all'avvio dell'applicazione che nei casi in cui la modalità a schermo intero potrebbe essere automaticamente disabilitata, ovvero quando l'applicazione perde e poi riacquisisce il "focus".

\begin{minted}[highlightlines={4,11}]{java}
@Override
public void onResume(Bundle savedInstanceState) {
    super.onCreate(savedInstanceState);
    goFullScreen();
}

@Override
public void onWindowFocusChanged(boolean hasFocus) {
    super.onWindowFocusChanged(hasFocus);
    if (hasFocus) {
        goFullScreen();
    }
}
\end{minted}


Questa soluzione ha l'effetto di nascondere le due barre di sistema, ma solo fintantoché l'utente non preme un qualsiasi punto dello schermo. Per questo Google ha introdotto anche una "modalità immersiva", che permette di mantenere l'applicazione a schermo intero fino a quando l'utente non scorre dai bordi dello schermo, come mostrato in seguito.\footnote{\url{https://developer.android.com/training/system-ui/immersive}}

\begin{figure}
\begin{minted}[highlightlines={7,9,10,11}]{java}
private void goFullScreen() {
    getWindow().getDecorView().setSystemUiVisibility(
        // Nasconde barra di navigazione e di stato
        View.SYSTEM_UI_FLAG_HIDE_NAVIGATION
        | View.SYSTEM_UI_FLAG_FULLSCREEN
        // Modalità immersiva
        View.SYSTEM_UI_FLAG_IMMERSIVE_STICKY
        // Evita lo spostamento del layout della pagina
        | View.SYSTEM_UI_FLAG_LAYOUT_STABLE
        | View.SYSTEM_UI_FLAG_LAYOUT_HIDE_NAVIGATION
        | View.SYSTEM_UI_FLAG_LAYOUT_FULLSCREEN);
}
\end{minted}
\end{figure}


\section{L'\emph{overscan} di sistema}
\label{sec:kiosk_overscan}

Come accennato, la modalità a schermo intero di Android permette comunque all'utente di usare le barre di sistema scorrendo dai lati dello schermo, e quindi potenzialmente di uscire dall'applicazione. In un sistema embedded questo non è desiderabile, ed esiste quindi un metodo alternativo per impedire che le barre di sistema vengano mostrate.

La soluzione prevede l'utilizzo della funzione \emph{overscan} del servizio di sistema \texttt{WindowManager}, il quale si occupa di gestire la visualizzazione delle finestre, la rotazione dello schermo, le animazioni, le transizioni, ecc.

L'overscan di sistema permette di modificare l'area di disegno delle finestre, o in altre parole i margini dello schermo. Impostando un margine negativo l'area di disegno dell'interfaccia di Android sarà estesa oltre l'area visibile dello schermo, con l'effetto di tagliare porzioni dell'interfaccia, come mostrato nella figura \ref{fig:overscan}.

\begin{figure}[h]
	\centering
	
	\scalebox{0.5} {
		\begin{tikzpicture}
		\filldraw[fill=white,draw=black]
		(0,0) rectangle (16,10);
		\filldraw[fill=black!10,draw=black]
		(0,1) rectangle (16,9) node[midway,text=black] {\Huge \textbf{AREA VISIBILE}};
		\end{tikzpicture}
	}

	\caption{Rappresentazione dell'overscan di sistema. La finestra sconfina l'area visibile.}
	\label{fig:overscan}
\end{figure}

Nell'esempio seguente viene utilizzato il comando \texttt{wm} (\texttt{WindowManager}) della shell di Android per impostare un margine negativo al lato superiore e inferiore dello schermo, in modo da nascondere barra di stato e di navigazione, alte in questo caso 25dp e 48dp\footnote{Density-independent Pixels, un'unità di misura che tiene in considerazione la densità dello schermo} (i valori possono essere diversi a seconda del dispositivo).

\begin{minted}[highlightlines=9]{bash}
> adb shell wm
usage: wm [subcommand] [options]
wm size [reset|WxH|WdpxHdp]
wm density [reset|DENSITY]
wm overscan [reset|LEFT,TOP,RIGHT,BOTTOM]
[...]

> adb shell wm overscan 0,-25,0,-48
\end{minted}

Una volta configurato l'overscan, l'impostazione dovrebbe essere memorizzata in modo permanente e sopravvivere al riavvio del sistema. Tuttavia, durante le sperimentazioni si è verificato almeno una volta che l'overscan non fosse ripristinato in automatico. Potrebbe quindi essere opportuno eseguire il comando ad ogni avvio del sistema, anche avviando un processo dall'applicazione stessa.

Per completezza, l'overscan può essere disabilitato con questo comando:

\begin{minted}{bash}
> adb shell wm overscan reset
\end{minted}


\section{La modalità "lock task"}
\label{sec:kiosk_locktask}

In aggiunta ai metodi illustrati nei capitoli precedenti, Android fornisce a partire dalla versione 5.0 una modalità chiamata \emph{lock task}, che fa parte di un insieme più ampio di API per la realizzazione di dispositivi dedicati (in passato chiamati COSU, Corporate-Owned Single-Use). Queste API fanno a loro volta parte di Android Enterprise, che propone soluzioni specifiche per casi d'uso aziendali.

Quando la modalità lock task viene abilitata, il sistema operativo entra in una modalità rigida che permette di usare solo un insieme di applicazioni definite manualmente tramite una \emph{whitelist}. Inoltre, la barra di stato viene disabilitata, le notifiche sono soppresse, e l'utilizzatore non può navigare nel sistema al di fuori delle app inserite in whitelist.\footnote{\url{https://developer.android.com/work/dpc/dedicated-devices/lock-task-mode}}

\begin{figure}[h]
	\centering
	\includegraphics[width=0.8\textwidth]{res/locktask.png}
	
	\caption{In modalità lock task la barra di stato e di navigazione sono disabilitate.}
	\label{fig:kiosk_locktask}
\end{figure}

Per implementare la modalità lock task sono necessari due componenti: un Device Policy Controller (DPC) e una \texttt{Activity} da lanciare. Il DPC deve essere un'applicazione "proprietaria del dispositivo", e cioè deve avere dei privilegi speciali per modificare alcune funzioni di sistema. Ha infatti il compito di configurare la modalità lock task elencando i nomi dei \emph{package} da inserire in whitelist.

La procedura per la realizzazione di un DPC è articolata ed è dettagliata nella documentazione di Android\footnote{\url{https://developer.android.com/guide/topics/admin/device-admin.html\#developing}}, ma è sufficiente sapere che viene fornito un metodo per scegliere quali applicazioni possono essere eseguite in modalità lock task:

\begin{minted}[highlightlines=4]{java}
DevicePolicyManager dpm =
    (DevicePolicyManager) getSystemService(Context.DEVICE_POLICY_SERVICE);
ComponentName adminName = getComponentName(this);
dpm.setLockTaskPackages(adminName, new String[] { "it.unitn.lode" });
\end{minted}

Google fornisce inoltre un'applicazione chiamata "Test DPC" che implementa numerose funzionalità legate alle policy aziendali, tra cui la configurazione della whitelist. L'applicazione è particolarmente utile durante lo sviluppo, perché permette di effettuare test rapidamente senza sviluppare un DPC. L'installazione è semplice e prevede di impostare "Test DPC" come proprietario del dispositivo:\footnote{\url{https://codelabs.developers.google.com/codelabs/cosu/index.html\#6}}

\begin{minted}{bash}
> adb install TestDPC.apk
> adb shell dpm set-device-owner com.afwsamples.testdpc/.DeviceAdminReceiver
\end{minted}

A questo punto, la \texttt{Activity} che vuole entrare in modalità lock task può semplicemente chiamare il metodo \texttt{startLockTask()}:

\begin{minted}[highlightlines=9]{java}
@Override
public void onResume() {
    super.onResume();
    
    DevicePolicyManager dpm =
        (DevicePolicyManager) getSystemService(Context.DEVICE_POLICY_SERVICE);

    if (dpm.isLockTaskPermitted(getPackageName())) {
        startLockTask();
    }
}
\end{minted}

Se ben configurata, questa soluzione, combinata con le tecniche mostrate nei capitoli precedenti, permette di avere un'applicazione a schermo intero da cui è impossibile uscire.

\section{L'avvio automatico dell'applicazione}
\label{sec:kiosk_launcher}

Una \texttt{Activity} può essere configurata per sostituire la schermata "home" di Android, ed essere quindi la prima ad essere mostrata all'avvio del sistema.

Il blocco di codice seguente mostra una porzione del file \texttt{AndroidManifest.xml}, tramite il quale è stato forzato il fatto che possa esistere una sola istanza alla volta di \texttt{MainActivity}, e che questa debba agire come attività "home" predefinita:

\begin{minted}[highlightlines={2,6,7}]{xml}
<activity android:name=".MainActivity"
          android:launchMode="singleTask">
    <intent-filter>
        <action android:name="android.intent.action.MAIN"/>
        <category android:name="android.intent.category.LAUNCHER"/>
        <category android:name="android.intent.category.DEFAULT" />
        <category android:name="android.intent.category.HOME" />
    </intent-filter>
</activity>
\end{minted}

\section{L'animazione di avvio}
\label{sec:kiosk_bootanimation}

Come ultimo aspetto, un requisito di un sistema embedded potrebbe essere quello di non mostrare i marchi di Android o del produttore dell'hardware. Fortunatamente, Android permette in modo abbastanza facile di sostituire l'animazione di avvio del sistema con una a piacere.\footnote{\url{https://android.googlesource.com/platform/frameworks/base/+/master/cmds/bootanimation/FORMAT.md}}

In fase di avvio, il sistema legge il file \texttt{/system/media/bootanimation.zip} ed estrae una descrizione dell'animazione (\texttt{desc.txt}) e un insieme di file PNG rappresentanti i fotogrammi da mostrare in sequenza su schermo.

Per fare un esempio, si prenda come riferimento questo file di descrizione:

\begin{minted}{text}
400 200 10
c 0 0 part0 #ffffff
\end{minted}

La prima riga indica parametri generali dell'animazione, in particolare la larghezza, l'altezza e il numero di fotogrammi al secondo da renderizzare.

La seconda riga indica invece un blocco di fotogrammi che devono essere eseguiti secondo precise regole. In particolare:

\begin{itemize}
	\item la lettera \texttt{c} indica che l'animazione verrà eseguita fino al suo completamento, anche nel caso in cui il sistema sia pronto prima;
	\item la prima occorrenza del numero \texttt{0} indica quante volte l'animazione deve essere ripetuta, in questo caso infinite volte, mentre la seconda il numero di fotogrammi di attesa prima della riproduzione del blocco successivo;
	\item \texttt{part0} è il nome della cartella in cui trovare la lista di file che compongono l'animazione;
	\item infine, l'ultimo parametro determina il colore di sfondo nel caso in cui l'animazione sia trasparente o non copra l'intero schermo.
\end{itemize}

I file ottenuti possono quindi essere inseriti in un archivio ZIP senza compressione, con un comando simile a questo:

\begin{minted}{bash}
zip -0qry -i \*.txt \*.png @ ../bootanimation.zip *.txt part*
\end{minted}

Infine, il file \texttt{bootanimation.zip} va caricato sul dispositivo Android tramite \texttt{adb} e la modalità debug con accesso root:

\begin{minted}{bash}
adb root
adb remount
adb push bootanimation.zip /system/media
adb reboot
\end{minted}


      \chapter{Sospensione della registrazione}
\label{cha:pausa}
Lorem ipsum dolor sit amet.


      \needspace{5\baselineskip}
\chapter{Sincronizzazione tra video e audio}
\label{cha:sync}

Nelle sezioni che seguono viene presentato il problema della sincronizzazione dei flussi video e audio, il cui effetto più evidente è lo sfasamento tra le parole e i movimenti del docente/relatore. I flussi sono acquisiti in modi diversi, per cui la sincronizzazione dell'inizio delle registrazioni non è banale. Viene quindi introdotta una soluzione che sfrutta un marcatore visivo per individuare un punto preciso e condiviso tra le registrazioni, in modo da poterle allineare.

\section{Definizione del problema}
\label{sec:sync_problema}

Come accennato, il problema sorge perché i tre flussi provengono da sorgenti e strumenti di acquisizione diversi. Si tratta in particolare di:

\begin{itemize}
	\item il video dell'ingresso HDMI, registrato con la classe \texttt{MediaRecorder} di Android o eventualmente con SDK dedicati. Nel primo caso si ha un metodo \texttt{start()} che si può prendere come istante approssimativo di inizio della registrazione, ma non è possibile ottenere il \emph{timestamp} preciso di inizio effettivo della registrazione (cioè quello del primo fotogramma);
	\item il video della videocamera IP, registrato con \texttt{ffmpeg} tramite il protocollo RTSP. La latenza di inizio registrazione è difficilmente stimabile, e cioè, in altre parole, dopo l'avvio del comando \texttt{ffmpeg} non è possibile sapere con precisione l'istante a cui corrisponde il primo fotogramma acquisito. La latenza è infatti influenzata da diversi fattori, tra cui la rete, i tempi di codifica, i buffer di trasmissione e di ricezione. Non è quindi facile trovare un intervallo di tempo abbastanza preciso per sincronizzare il video con gli altri flussi;
	\item l'audio di un microfono, collegato via cavo direttamente alla board Android. Questo può essere registrato con la stessa istanza di \texttt{MediaRecorder} del primo punto, ottenendo quindi una sincronizzazione naturale, almeno a livello teorico.
\end{itemize}

A questo punto si ha che le due sorgenti acquisite via cavo (HDMI e audio) sono tra loro sincronizzate, mentre resta il problema di allineare il video RTSP con la traccia audio. Uno sfasamento di anche solo 200-300 millisecondi risulta infatti percepibile e può rendere fastidiosa la fruizione del video.

\section{Proposta di soluzione}
\label{sec:sync_soluzione}

Con le premesse della sezione precedente, una soluzione percorribile è la sincronizzazione tra il video HDMI e il video RTSP, in modo da ottenere transitivamente una sincronizzazione anche con l'audio.

Questo obiettivo è raggiungibile con una ragionevole precisione utilizzando un riferimento visivo che possa essere proiettato su schermo e quindi catturato dalla videocamera. Questo riferimento dovrebbe essere rilevabile in modo automatizzato, in modo da svolgere la funzione di "ciak". Nella pratica potrebbe trattarsi di una schermata verde mostrata per un secondo dall'applicazione, in modo simile alle schermate monocolore utilizzate spesso dai proiettori in caso di assenza di segnale.

La figura \ref{fig:sync1} mostra su una linea del tempo gli intervalli delle registrazioni RTSP e HDMI. Supponiamo in questo caso che la registrazione della videocamera parta per prima, e che la linea verticale indichi l'istante temporale in cui il marcatore visivo viene catturato dalla videocamera. Potrebbe quindi trattarsi del momento in cui la schermata verde compare oppure scompare dalla proiezione.

\begin{figure}[H]
	\centering
	
	\begin{tikzpicture}
	
	\pgfdeclarelayer{fg}
	\pgfsetlayers{main,fg}
	
	\begin{pgfonlayer}{fg}
	
		\filldraw[fill=black!5,draw=black] (0,0) rectangle ++(8,0.5);
		\node[right] at(8,0.25) {Video RTSP};
		
		\filldraw[fill=black!5,draw=black] (4,-0.7) rectangle ++(4,0.5);
		\node[right] at(8,-0.45) {Video HDMI};
		
		\draw[ultra thick] (2,0.75) -- (2,-1);
	
	\end{pgfonlayer}
	
	\end{tikzpicture}

	\caption{}
	\label{fig:sync1}
\end{figure}

Quello che ci interessa ottenere ora è che le due registrazioni video inizino nello stesso istante della linea del tempo. Dobbiamo quindi tagliare un pezzo del video RTSP in modo che il primo fotogramma RTSP corrisponda al primo fotogramma HDMI. Per farlo ci servono due intervalli di tempo, indicati come A e B nella figura \ref{fig:sync2}.

\begin{figure}[H]
	\centering
	
	\begin{tikzpicture}
	
	\pgfdeclarelayer{fg}
	\pgfsetlayers{main,fg}
	
	\begin{pgfonlayer}{fg}
	
	\filldraw[fill=black!5,draw=black] (0,0) rectangle ++(8,0.5);
	\node[right] at(8,0.25) {Video RTSP};
	
	\filldraw[fill=black!5,draw=black] (4,-0.7) rectangle ++(4,0.5);
	\node[right] at(8,-0.45) {Video HDMI};
	
	\filldraw[fill=black!25,draw=black] (2,-0.7) rectangle ++(2,0.5) node[midway] {B};
	\filldraw[fill=black!25,draw=black] (0,-0.7) rectangle ++(2,0.5) node[midway] {A};
	
	\draw[ultra thick] (2,0.75) -- (2,-1);
	
	\end{pgfonlayer}
	
	\end{tikzpicture}
	
	\caption{}
	\label{fig:sync2}
\end{figure}

Ottenere l'intervallo B è abbastanza facile, trattandosi di una sottrazione tra \emph{timestamp}. Siamo infatti a conoscenza sia del \emph{timestamp} di avvio della registrazione HDMI sia di quello del riferimento visivo (la linea verticale). L'intervallo B è quindi ottenibile con relativa precisione calcolando la sottrazione tra questi due valori.

L'intervallo A è invece equivalente al tempo trascorso tra l'inizio della registrazione RTSP e l'istante in cui il riferimento visivo compare su schermo. Questo deve essere ricavato analizzando il video tramite strumenti di \emph{computer vision}, come mostrato nella sezione \ref{sec:sync_impl}.

Ottenuti gli intervalli A e B, la loro somma determina lo sfasamento tra video RTSP e HDMI. La rimozione di $(A+B)$ secondi dall'inizio del video RTSP permette quindi di sincronizzare i due flussi video e di conseguenza anche l'audio.

\section{Implementazione e sperimentazione}
\label{sec:sync_impl}

Il primo passaggio per implementare la soluzione è predisporre la visualizzazione del "ciak". Come accennato può trattarsi di una schermata monocolore a schermo intero, ottenibile su Android tramite una \texttt{View} con uno sfondo colorato:

\begin{minted}[highlightlines={8-12}]{xml}
<?xml version="1.0" encoding="utf-8" ?>
<FrameLayout xmlns:android="http://schemas.android.com/apk/res/android"
             xmlns:tools="http://schemas.android.com/tools"
             android:layout_width="match_parent"
             android:layout_height="match_parent"
             tools:context=".MainActivity">

    <View android:id="@+id/rect"
          android:visibility="GONE"
          android:layout_width="match_parent"
          android:layout_height="match_parent"
          android:background="#00ff00" />

</FrameLayout>
\end{minted}

La \texttt{View} è inizialmente nascosta (\texttt{GONE}) e può essere resa visibile via codice così:

\begin{minted}{java}
findViewById(R.id.rect).setVisibility(View.VISIBLE);
\end{minted}

A questo punto è necessario fare uso di timer e ritardi appositamente studiati per ottenere la situazione dello schema della figura \ref{fig:sync1}, cioè fare in modo che la videocamera riprenda il momento in cui la schermata verde viene abilitata.

Passando alla fase di post-elaborazione del materiale acquisito, l'obiettivo principale è riuscire a ricavare dalla registrazione RTSP il tempo in cui compare il marcatore visivo, e cioè l'intervallo A della figura \ref{fig:sync2}. Per fare ciò è possibile usare \texttt{OpenCV}, una nota libreria di visione artificiale.

Più in dettaglio, la tecnica sperimentata consiste nel calcolare la media delle tre componenti di colore RGB per ciascun fotogramma, e di provare poi a rilevare le variazioni significative di verde. Concettualmente, il parametro da considerare per individuare le variazioni è la "distanza" tra il colore verde e le altre componenti di colore, e cioè, in formula, $G - \frac{R+B}{2}$. Il blocco di codice che segue calcola questa distanza per ogni fotogramma del video. 

\inputminted[xleftmargin=\parindent,linenos]{python}{res/opencv.py}

Per comprendere meglio cosa sta succedendo, conviene usare una rappresentazione grafica dei dati raccolti. La figura \ref{fig:sync_opencv1} mostra la variazione nel tempo delle tre coordinate RGB di un video in cui al secondo 6 gran parte dello schermo ripreso diventa verde (figura \ref{fig:sync_video}). Dal grafico è certamente evidente che al secondo 6 c'è un cambiamento significativo e che il valore del verde aumenta leggermente, ma se si osserva meglio non risulta in realtà così facile individuare con precisione e in modo automatizzato l'istante di inizio e fine della schermata verde.

\begin{figure}[htbp]
	\centering
	
	\includegraphics{res/opencv_channels_tight.pdf}
	
	\caption{Variazione delle componenti colore RGB in relazione al tempo.}
	\label{fig:sync_opencv1}
\end{figure}

\begin{figure}[htbp]
	\centering
	
	\begin{subfigure}[t]{0.5\textwidth}
		\centering
		\includegraphics[width=\textwidth]{res/opencv1.png}
		\caption{$t=0s$}
	\end{subfigure}%
	~ 
	\begin{subfigure}[t]{0.5\textwidth}
		\centering
		\includegraphics[width=\textwidth]{res/opencv2.png}
		\caption{$t=6s$}
	\end{subfigure}

	\caption{Due fotogrammi del video analizzato, rispettivamente il primo fotogramma e un fotogramma di poco successivo al secondo 6.}
	\label{fig:sync_video}
\end{figure}

La figura \ref{fig:sync_opencv2} mostra invece la distanza tra la componente verde e la media delle altre componenti, come calcolato alla riga 15 dello script Python. Qua il salto è molto più facile da individuare, anche in modo automatizzato, perché si tratta di un delta di quasi 20 che avviene in modo repentino dopo un intervallo di stabilità.

\begin{figure}[htbp]
	\centering
	
	\includegraphics{res/opencv_diff.pdf}
	
	\caption{Variazione della variabile \texttt{diff}, cioè la distanza tra componente verde e la media delle altre componenti, in relazione al tempo.}
	\label{fig:sync_opencv2}
\end{figure}

Per la precisione, la variazione in questo caso avviene leggermente prima del secondo 6, come si evince con evidenza dall'output dello script:

\begin{minted}[highlightlines=5]{text}
[...]
5.747986463620982 => -7.589908854166694
5.81405527354766 => -7.586672634548577
5.880124083474336 => -7.579766167534601
5.946192893401014 => 11.272211914062495
6.012261703327693 => 12.108198242187711
6.078330513254371 => 12.092488064236306
6.144399323181049 => 12.112571614583288
6.210468133107726 => 12.133828125000036
[...]
\end{minted}

È quindi $5,94$ secondi il valore corrispondente all'intervallo A dello schema \ref{fig:sync2}, che può essere sommato a B per ottenere l'intervallo di tempo da tagliare all'inizio del video RTSP.

Supponendo che $A+B$ dia come risultato $8$, possiamo ora ritagliare il video RTSP e ottenere un ragionevole sincronismo tra i due flussi video. Il blocco che segue è una versione minima di un comando che usa \texttt{ffmpeg} per rimuovere i primi 8 secondi del file \texttt{rtsp.mp4}.

\begin{minted}[escapeinside=||]{text}
> ffmpeg |\colorbox{LightCyan}{-ss 8}| -i rtsp.mp4 -r 25 rtsp-synced.mp4 -y
\end{minted}

I risultati ottenuti con questa soluzione sono soddisfacenti, ma soffrono di uno sfasamento cronico dovuto al fatto che il \emph{timestamp} memorizzato alla chiamata del metodo \texttt{start()} per l'avvio della registrazione HDMI non è preciso. Questo dipende dal fatto che il metodo \texttt{start()} ritorna istantaneamente, ma la registrazione effettiva inizia in realtà qualche decina (?) di millisecondi dopo. Considerato che questo ritardo è consistente e l'hardware è fisso, una possibilità potrebbe essere di compensarlo manualmente nel calcolo dell'intervallo $A+B$.

\hl{Questo paragrafo è da quantificare in numeri appena recupero i dati dei test, che ora non ho a portata di mano...}


      \chapter{Rilevamento delle differenze tra fotogrammi}
\label{cha:diff}

Una funzionalità innovativa introdotta dal sistema LODE è la possibilità per gli studenti di acquisire in tempo reale durante le lezioni degli screenshot di quanto viene proiettato, utilizzando un'apposita applicazione web. Dal punto di vista tecnico questo significa che il "box" deve poter acquisire singolarmente dei fotogrammi dall'ingresso HDMI e inviarli al server.

Per evitare che richieste di screenshot molto vicine provochino inutili upload di screenshot, è possibile implementare un sistema per rilevare se uno screenshot è effettivamente nuovo, e in caso negativo riusare quello precedente.

\section{Il formato YUV420SP}
\label{sec:diff_yuv}

Prima di iniziare a lavorare sul confronto delle immagini, è opportuno familiarizzare con il formato YUV420SP, che si incontra frequentemente nello sviluppo Android con il nome di NV12 o NV21.

Prima di tutto, YUV420 è un termine approssimativo per riferirsi a Y'CbCr 4:2:0 \cite{yuv}, un formato di pixel appartenente alla famiglia Y'CbCr, in cui il colore di un pixel è scomposto nelle componenti Y' (luminanza, scala di grigi\footnotemark{}), Cb/U (proiezione del blu) e Cr/V (proiezione del rosso).

\footnotetext{La suddivisione delle componenti in Y, U e V deriva dal mondo analogico, quando c'era la necessità di supportare sia tv in bianco e nero che a colori. Avere una componente in scala di grigi isolata permette infatti di mantenere la retrocompatibilità pur introducendo il colore. In ambito digitale, è più corretto usare il termine Y'CbCr anziché YUV.}

Questa divisione permette di applicare una tecnica chiamata sottocampionamento della crominanza, che riduce la risoluzione delle due componenti della crominanza (Cb e Cr), lasciando a piena risoluzione la luminanza. Questa tecnica è fondata sul fatto che l'occhio umano è molto più sensibile all'intensità di luce che al colore (figura \ref{fig:diff_mit}), per cui anche se riduciamo i "bit" per la rappresentazione della crominanza la differenza è nella gran parte dei casi quasi impercettibile \cite{luminance}. La grandissima maggioranza dei video normalmente fruibili tramite Internet o la televisione digitale sono codificati con un formato dei pixel Y'CbCr 4:2:0, che è il più diffuso in ambiti non professionali.

% https://twitter.com/AkiyoshiKitaoka/status/1028473566193315841

\begin{figure}[htbp]
	\centering
	
	\begin{subfigure}[t]{0.45\textwidth}
		\centering
		\includegraphics[width=\textwidth]{res/mit1.png}
	\end{subfigure}%
	~ 
	\begin{subfigure}[t]{0.45\textwidth}
		\centering
		\includegraphics[width=\textwidth]{res/mit2.png}
	\end{subfigure}
	
	\caption{Rappresentazione grafica realizzata dal MIT\protect\footnotemark{} per mostrare come l'occhio umano è molto sensibile all'intensità della luce (luminanza). L'illusione ottica fa credere che i due riquadri A e B siano di sfumature di grigio diverse, quando in realtà sono identici.}
	\label{fig:diff_mit}
\end{figure}

\footnotetext{\url{http://persci.mit.edu/gallery/checkershadow}}

La figura \ref{fig:diff_yuv420} mostra il sottocampionamento della crominanza 4:2:0 applicato a una griglia di dimensione 4 x 2 pixel. La componente Y, cioè la luminanza, viene campionata a piena risoluzione, e cioè per ogni pixel vengono catturate informazioni piene. L'informazione sulla crominanza, composta dalla componente blu e rossa\footnotemark{}, viene invece catturata a $1/4$ della risoluzione, e cioè è condivisa tra 4 pixel. In confronto a RGB24, questo formato richiede in media 12 bit per pixel anziché 24 (da qui il nome "NV12"), con un risparmio di dati del 50\% senza sacrificare visibilmente la qualità complessiva.

\footnotetext{L'informazione sul verde non è esplicitamente memorizzata ma rappresenta circa il 60\% della componente Y:\\ \url{https://news.ycombinator.com/item?id=1892248}}

\begin{figure}[htbp]
	\centering
	
	\includegraphics[width=0.8\textwidth]{res/yuv420.pdf}
%	\includegraphics[width=0.9\textwidth]{res/yuv420.png}
	
	\caption{Sottocampionamento della crominanza 4:2:0. Il 4 indica la larghezza della griglia, che ha altezza 2 (fissa). Il 2 indica che la risoluzione orizzontale è dimezzata (2 campioni), mentre lo zero che non ci sono campioni diversi tra la prima e la seconda riga.}
	\label{fig:diff_yuv420}
\end{figure}

Nel momento in cui YUV420 deve essere rappresentato sotto forma di bit, i tre piani (Y, U e V) vengono isolati. Ad esempio, un pixel verrebbe rappresentato così (ogni lettera è un bit) \cite{vlc}:

\definecolor{ChromaBlue}{HTML}{1160c6}
\definecolor{ChromaRed}{HTML}{E60000}
\definecolor{ChromaSP}{HTML}{ff9c00}

\tikzset{
	pixel/.style={node distance=1pt,font=\sffamily},
	pixelY/.style={pixel,fill=black!15},
	pixelU/.style={pixel,fill=ChromaBlue,text=white},
	pixelV/.style={pixel,fill=ChromaRed,text=white},
	pixelSP/.style={pixel,fill=ChromaSP,text=white},
}

\begin{figure}[H]
	\begin{tikzpicture}
	\node[pixelY] (y) {YYYYYYYY};
	\node[pixelU,right=of y] (u) {UU};
	\node[pixelV,right=of u] {VV};
	\end{tikzpicture}
\end{figure}

Mentre in RGB ciascun pixel richiederebbe 8 bit per canale, in YUV420 i due piani della crominanza (blu e rosso) ne richiedono solo $1/4$, cioè 2.

Di conseguenza, 2 pixel sarebbero rappresentati così, dove ciascuna coppia di U/V è relativa a un pixel:

\begin{figure}[H]
	\begin{tikzpicture}
	\node[pixelY] (y) {YYYYYYYY YYYYYYYY};
	\node[pixelU,right=of y] (u) {UU UU};
	\node[pixelV,right=of u] {VV VV};
	\end{tikzpicture}
\end{figure}

Il formato YUV420SP è invece una variante di YUV420 \emph{Semi Planar}, che significa che i piani U e V sono intrecciati in un unico piano. Un pixel YUV420SP (NV12) verrebbe quindi rappresentato così:

\begin{figure}[H]
	\begin{tikzpicture}
	\node[pixelY] (y) {YYYYYYYY};
	\node[pixelSP,right=of y] {UVUV};
	\end{tikzpicture}
\end{figure}

E 2 pixel così:

\begin{figure}[H]
	\begin{tikzpicture}
	\node[pixelY] (y) {YYYYYYYY YYYYYYYY};
	\node[pixelSP,right=of y] {UVUV UVUV};
	\end{tikzpicture}
\end{figure}

Come accennato, esiste anche il formato NV21, supportato da Android, che è uguale a NV12 ma con la differenza che le componenti U e V all'interno del secondo piano sono scambiate, come mostrato nell'esempio:

\begin{figure}[H]
	\begin{tikzpicture}
	\node[pixelY] (y) {YYYYYYYY YYYYYYYY};
	\node[pixelSP,right=of y] {VUVU VUVU};
	\end{tikzpicture}
\end{figure}

Come si nota dagli schemi, in NV12/NV21 le informazioni sulla luminanza sono raggruppate all'inizio e utilizzano 1 byte per pixel. Questa caratteristica tornerà utile nelle prossime sezioni, in combinazione al fatto che Android fornisce le immagini acquisite dall'ingresso HDMI anche in formato NV21.

\section{Confronto pixel per pixel}
\label{sec:diff_full}

La prima possibilità esplorata è la più immediata: catturare i fotogrammi in formato "non compresso" (NV21) e confrontarli byte per byte. La prima coppia di byte che ha valori diversi determina l'esistenza di una differenza tra i due fotogrammi.

Ricordando la classe \texttt{Camera} di Android introdotta nella sezione \ref{sec:hdmi_camera1}, si può sfruttare il metodo \texttt{setPreviewCallback(...)} per ricevere i dati "raw" che vengono mostrati su schermo.\footnote{\url{https://developer.android.com/reference/android/hardware/Camera.PreviewCallback.html}}

\begin{minted}{java}
camera.setPreviewCallback(new PreviewCallback() {
    @Override
    public void onPreviewFrame(byte[] data, Camera camera) {
        // `data` contiene i dati in formato NV21
    }
});
\end{minted}

Supponendo ora di avere due array di byte contenenti due fotogrammi catturati in momenti diversi, si può concludere che se le due immagini sono identiche anche i valori dei pixel e quindi i relativi "byte" saranno identici. Di conseguenza, per confrontare i due array si può usare il metodo \texttt{Arrays.equals(arr1, arr2)} contenuto in \texttt{java.util}, che è implementato con un ciclo che confronta gli array byte per byte e si ferma eventualmente alla prima differenza.\footnote{\url{https://hg.openjdk.java.net/jdk8/jdk8/jdk/file/687fd7c7986d/src/share/classes/java/util/Arrays.java\#l2668}}

Questo metodo di rilevamento delle differenze è stato verificato e funziona in modo affidabile, ma ha lo svantaggio non trascurabile di non essere molto performante. Su uno degli hardware testati, il confronto di due fotogrammi con risoluzione 1920 x 1080 (e quindi di $3\,110\,400$ byte\footnote{$(1920 \cdot 1080) / 2$}) richiedeva in media tra i 400 e i 450 millisecondi. Su un altra board simile il tempo medio risultava di circa 100 millisecondi.

Come accennato nella sezione \ref{sec:diff_yuv}, l'occhio umano è più sensibile alla luminanza che alla crominanza. Si può quindi tentare di ridurre la quantità di byte da confrontare selezionando soltanto il piano Y, che rappresenta la luminanza.

In questo caso il confronto va implementato manualmente per considerare soltanto una parte dell'array, quella corrispondente al primo piano, che qua si suppone utilizzi un byte per pixel come nel caso di NV12/NV21:

\begin{minted}{java}
public boolean isLuminanceEqual(byte[] arr1, byte[] arr2, int width, int height) {
    int maxLength = width * height;
    
    for (int i = 0; i < maxLength; i++) {
        if (arr1[i] != arr2[i]) {
            return false;
        }
    }
    
    return true;
}
\end{minted}

Assumendo una risoluzione di 1920 x 1080 pixel, il numero di byte da verificare si riduce a $2\,073\,600$, cioè $2/3$ del totale. La riduzione è interessante ma non risolutiva in termini di prestazioni.

\section{Confronto parziale}
\label{sec:diff_parziale}

Una soluzione alternativa a confrontare l'intera immagine è individuare un sottoinsieme di pixel, distribuito in modo da rendere sufficientemente probabile il rilevamento delle differenze.

Si parte dal presupposto che durante una lezione vengano proiettati contenuti le cui variazioni sono abbastanza evidenti. Si pensi ad esempio a una presentazione di slide dove con il passaggio da una pagina all'altra varia l'intero contenuto proiettato o viene aggiunta una riga di testo, oppure a una finestra del browser dove lo scorrimento della pagina provoca lo spostamento di tutto il contenuto.

Si può quindi costruire una griglia virtuale i cui punti di intersezione rappresentano i pixel da confrontare, come mostrato in figura \ref{fig:diff_slide1}. L'approccio funziona bene in molti casi: prendendo come riferimento la figura si immagini come l'aggiunta di una riga verrebbe facilmente rilevata dai punti stabiliti (che sono fissi).

\tikzset{
	gridline/.style={line width=0.4pt,color=lightgray},
	gridarrow/.style={-{Straight Barb},very thick,red},
}

\begin{figure}[htbp]
	\centering

	\begin{tikzpicture}
	% https://tex.stackexchange.com/questions/9559/drawing-on-an-image-with-tikz

	\node[anchor=south west,inner sep=0] (image) at (0,0)
		{\includegraphics[width=0.9\textwidth]{res/test-slide/beamer.pdf}};
	
	\begin{scope}[x={(image.south east)},y={(image.north west)}]
%	\draw[color=black,thick,xstep=.1,ystep=.1] (0,0) grid (1,1);
	\foreach \x in {1,...,9} { \draw[gridline] (\x/10,0) -- (\x/10,1); }
	\foreach \y in {1,...,49} { \draw[gridline] (0,\y/50) -- (1,\y/50); }
	\end{scope}
	
	\end{tikzpicture}
	
	\caption{La griglia è composta da 9 righe orizzontali e 49 verticali, per un totale di 441 punti di intersezione.}
	\label{fig:diff_slide1}
\end{figure}

Ci sono tuttavia altri casi in cui i cambiamenti potrebbero riguardare delle aree non rilevate dai punti selezionati, come si deduce dalla figura \ref{fig:diff_slide2}.

\begin{figure}[htbp]
	\centering
	
	\begin{subfigure}[t]{0.5\textwidth}
		\begin{tikzpicture}
		\node[anchor=south west,inner sep=0] (image) at (0,0)
		{\includegraphics[width=\textwidth]{res/slide3.png}};
		
		\begin{scope}[x={(image.south east)},y={(image.north west)}]
		\foreach \x in {1,...,9} { \draw[gridline] (\x/10,0) -- (\x/10,1); }
		\foreach \y in {1,...,49} { \draw[gridline] (0,\y/50) -- (1,\y/50); }
		
		\draw[gridarrow] (0.62,0.15) -- (0.68,0.29);
		\end{scope}
		
		\end{tikzpicture}
	\end{subfigure}%
	~ 
	\begin{subfigure}[t]{0.5\textwidth}
		\begin{tikzpicture}
		\node[anchor=south west,inner sep=0] (image) at (0,0)
		{\includegraphics[width=\textwidth]{res/slide4.png}};
		
		\begin{scope}[x={(image.south east)},y={(image.north west)}]
		\foreach \x in {1,...,9} { \draw[gridline] (\x/10,0) -- (\x/10,1); }
		\foreach \y in {1,...,49} { \draw[gridline] (0,\y/50) -- (1,\y/50); }
		
		\draw[gridarrow] (0.62,0.15) -- (0.68,0.29);
		\end{scope}
		
		\end{tikzpicture}
	\end{subfigure}
	
	\caption{Anche in questo caso i punti di intersezione sono 441. Si noti però che l'area puntata dalla freccia presenta delle differenze tra i due fotogrammi che non vengono rilevate. Le immagini sono state gentilmente offerte dal professor Alberto Montresor.}
	\label{fig:diff_slide2}
\end{figure}

Si possono quindi considerare delle strategie alternative che permettono di aggirare questo problema. Ad esempio:

\begin{itemize}
	\item il numero di colonne può essere aumentato in modo da rendere la griglia più "fitta" e rilevare non solo aggiunte di frasi ma anche brevi parole o simboli;
	\item il numero di righe e colonne della griglia può essere ricalcolato ad ogni controllo aggiungendo un fattore di casualità. Se un controllo non riesce a rilevare differenze, la volta successiva potrebbe invece riuscirci, dato che i pixel campione saranno diversi (con ragionevole probabilità);
	\item la distanza orizzontale e verticale tra i punti può essere mantenuta costante, ma si può invece aggiungere uno scorrimento graduale verso destra e verso il basso dei punti, in modo da coprire aree diverse ad ogni controllo;
	\item la strategia di scegliere un numero casuale di righe e colonne può essere resa più intelligente, per evitare che la stessa coordinata orizzontale e/o verticale venga considerata in diverse combinazioni della griglia.
\end{itemize}

La strategia del secondo punto è stata sperimentata su diversi contenuti e ha dato ottimi risultati. Gran parte dei cambiamenti venivano rilevati al primo controllo, e in ogni caso entro 3-4 controlli. Nei test l'intervallo di verifica delle differenze era 500 millisecondi, per cui il ritardo massimo di rilevamento risultava di circa 2 secondi.

Nei test effettuati il numero di colonne e righe della griglia veniva scelto in modo casuale ad ogni controllo, scegliendo rispettivamente dall'intervallo $[10, 20]$ e $[60, 80]$. Con questi dati, nel peggiore dei casi il numero di pixel/byte da verificare è $1\,600$, un valore estremamente limitato se confrontato con i 3 milioni di byte della sezione \ref{sec:diff_full}. Le prestazioni di questa soluzione sono risultate di conseguenza ottime, con un tempo di confronto solitamente non superiore a 1-2 millisecondi.

\section{Estrazione slide in post-elaborazione}
\label{sec:diff_postprod}

Un'altra possibile applicazione del rilevamento delle differenze è l'estrazione di "istantanee" dal video registrato, in fase di post-elaborazione. I fotogrammi estratti potrebbe poi essere utilizzati come indici legati a punti specifici del video, consentendo di costruire funzionalità come il salto preciso a una slide, oppure in generale come anteprime rappresentative del video.

Per estrarre tutti i fotogrammi "unici" da un video ci viene ancora una volta in aiuto \texttt{ffmpeg}, in particolare con un filtro chiamato \texttt{freezedetect} introdotto nella versione 4.2 (agosto 2019).

Il filtro \texttt{freezedetect} prevede due opzioni \cite{ffmpeg}:

\begin{itemize}
	\item \texttt{noise}: indica la soglia di rumore sopra la quale due fotogrammi vengono considerati diversi. Questa soglia può essere specificata in decibel (aggiungendo dB al valore), oppure con un numero decimale nell'intervallo $[0, 1]$. Il valore predefinito è $-60 dB$, equivalente a $0,001$;
	\item \texttt{duration}: l'intervallo di tempo oltre il quale un fotogramma viene considerato in stato di "freeze", e quindi invariato secondo il parametro \texttt{noise}. Il valore predefinito è 2 secondi.
\end{itemize}

Il comando che segue configura il filtro \texttt{freezedetect} con una soglia di rumore pari a $0,01$ e una durata minima di 5 secondi. Questi parametri sono stati ricavati empiricamente utilizzando come input dei video catturati durante reali lezioni universitarie. L'intervallo di 5 secondi è in particolare pensato per evitare che cambi momentanei della schermata vengano considerati rilevanti.

\begin{minted}{bash}
> ffmpeg -i input.mp4 \
         -vf "freezedetect=noise=0.01:duration=5,metadata=print:file=log.txt" \
         -an -f null -
\end{minted}

Nel comando si può notare che oltre a \texttt{freezedetect} viene applicato un secondo filtro chiamato \texttt{metadata}. Questo filtro si occupa di leggere i valori prodotti dal filtro \texttt{freezedetect} e di scriverli in un file chiamato \texttt{log.txt}.

I metadati prodotti sono identificati da tre chiavi, in particolare:

\begin{itemize}
	\item \texttt{lavfi.freezedetect.freeze\_start}: indica il \emph{timestamp} del primo fotogramma in cui è stato individuato il freeze (inclusi i primi secondi di \texttt{duration});
	\item \texttt{lavfi.freezedetect.freeze\_duration}: indica in secondi l'intervallo totale durante il quale il fotogramma è rimasto invariato;
	\item \texttt{lavfi.freezedetect.freeze\_end}: indica il \emph{timestamp} dell'ultimo fotogramma dell'intervallo.
\end{itemize}

Dopo aver eseguito il comando, il file \texttt{log.txt} conterrà i dati raccolti. Nell'esempio che segue sono state rilevate 3 istantanee, ai secondi 0, 132 e 200.

\begin{figure}[H]
\begin{minted}[highlightlines={2,7,12}]{text}
frame:192  pts:451200  pts_time:5.01333
lavfi.freezedetect.freeze_start=0
frame:5416 pts:11594764 pts_time:128.831
lavfi.freezedetect.freeze_duration=128.831
lavfi.freezedetect.freeze_end=128.831
frame:5764 pts:12334264 pts_time:137.047
lavfi.freezedetect.freeze_start=132.031
frame:8439 pts:18041731 pts_time:200.464
lavfi.freezedetect.freeze_duration=68.433
lavfi.freezedetect.freeze_end=200.464
frame:8648 pts:18493226 pts_time:205.48
lavfi.freezedetect.freeze_start=200.464
\end{minted}
\caption{Contenuto del file di output \texttt{log.txt}.}
\label{fig:diff_freezeout}
\end{figure}

Se si vogliono ora estrarre i fotogrammi individuati, ci sono due strade. La prima è quella di ritagliare più volte i primi N secondi del video (con \texttt{-ss N}) e poi estrarre il primo fotogramma immediatamente successivo (\texttt{-frames:v 1}):

\begin{minted}{bash}
> ffmpeg -ss 1 -i input.mp4 \
         -ss 133 -i input.mp4 \
         -ss 201 -i input.mp4 \
         -map 0:v -frames:v 1 out1.png \
         -map 1:v -frames:v 1 out2.png \
         -map 2:v -frames:v 1 out3.png -y
\end{minted}

Si osservi che l'offset in secondi è stato incrementato di uno e arrotondato per difetto, in modo da evitare eventuali artefatti di codifica dovuti al cambio di scena.

La soluzione precedente ha lo svantaggio di caricare più volte il file in parallelo, causando un uso di memoria RAM molto elevato nel caso in cui i fotogrammi da estrarre siano tanti (più di qualche decina).

L'alternativa è quindi usare il filtro \texttt{select}, che permette di selezionare solo determinati fotogrammi dall'input e di passarli all'output. Questo metodo fa un uso molto limitato di memoria ma è anche molto più lento, perché non sfrutta la funzionalità di seeking rapido offerta dall'opzione \texttt{-ss}.

Sono inoltre richiesti gli indici dei fotogrammi, anziché i relativi \emph{timestamp}, che vanno quindi estratti dall'output mostrato nella figura \ref{fig:diff_freezeout}. In compenso non è però necessario aggiungere un offset ai valori, perché gli indici dei fotogrammi sono già relativi a un momento successivo di 5 secondi rispetto all'inizio del freeze.

\begin{minted}{bash}
> ffmpeg -i input.mp4 \
         -vf "select='eq(n,192)+eq(n,5764)+eq(n,8648)'" \
         -vsync 0 %d.png
\end{minted}

In questo caso c'è da notare l'aggiunta dell'opzione \texttt{-vsync 0}, che ha lo scopo di disabilitare la duplicazione dei fotogrammi che viene normalmente applicata per simulare un framerate costante.

A questo punto un possibile uso dei file generati è la generazione di un mosaico, con la possibilità di configurare la dimensione delle singole immagini e il numero di colonne e righe della griglia:

\begin{minted}{bash}
> ffmpeg -i %d.png -vf "scale=200:-1,tile=5x3" \
         -frames:v 1 tiles.png -y
\end{minted}


      \chapter{Acquisizione audio}
\label{cha:audio}
Lorem ipsum dolor sit amet.


      \chapter{Conclusione}
\label{cha:conclusione}
Lorem ipsum dolor sit amet.


      
    \endgroup


    % bibliografia in formato bibtex
    %
    % aggiunta del capitolo nell'indice
    \addcontentsline{toc}{chapter}{Bibliografia}
    % stile con ordinamento alfabetico in funzione degli autori
    \bibliographystyle{plain}
    \bibliography{parti/biblio}
%%%%%%%%%%%%%%%%%%%%%%%%%%%%%%%%%%%%%%%%%%%%%%%%%%%%%%%%%%%%%%%%%%%%%%%%%%
%%%%%%%%%%%%%%%%%%%%%%%%%%%%%%%%%%%%%%%%%%%%%%%%%%%%%%%%%%%%%%%%%%%%%%%%%%
%% Nota
%%%%%%%%%%%%%%%%%%%%%%%%%%%%%%%%%%%%%%%%%%%%%%%%%%%%%%%%%%%%%%%%%%%%%%%%%%
%% Nella bibliografia devono essere riportati tutte le fonti consultate 
%% per lo svolgimento della tesi. La bibliografia deve essere redatta 
%% in ordine alfabetico sul cognome del primo autore. 
%% 
%% La forma della citazione bibliografica va inserita secondo la fonte utilizzata:
%% 
%% LIBRI
%% Cognome e iniziale del nome autore/autori, la data di edizione, titolo, casa editrice, eventuale numero dell’edizione. 
%% 
%% ARTICOLI DI RIVISTA
%% Cognome e iniziale del nome autore/autori, titolo articolo, titolo rivista, volume, numero, numero di pagine.
%% 
%% ARTICOLI DI CONFERENZA
%% Cognome e iniziale del nome autore/autori (anno), titolo articolo, titolo conferenza, luogo della conferenza (città e paese), date della conferenza, numero di pagine. 
%% 
%% SITOGRAFIA
%% La sitografia contiene un elenco di indirizzi Web consultati e disposti in ordine alfabetico. 
%% E’ necessario:
%%   Copiare la URL (l’indirizzo web) specifica della pagina consultata
%%   Se disponibile, indicare il cognome e nome dell’autore, il titolo ed eventuale sottotitolo del testo
%%   Se disponibile, inserire la data di ultima consultazione della risorsa (gg/mm/aaaa).    
%%%%%%%%%%%%%%%%%%%%%%%%%%%%%%%%%%%%%%%%%%%%%%%%%%%%%%%%%%%%%%%%%%%%%%%%%%
%%%%%%%%%%%%%%%%%%%%%%%%%%%%%%%%%%%%%%%%%%%%%%%%%%%%%%%%%%%%%%%%%%%%%%%%%%
    

    \titleformat{\chapter}
        {\normalfont\Huge\bfseries}{Allegato \thechapter}{1em}{}
    \appendix
    \chapter{La libreria \texttt{MobileFFmpeg}}
\label{cha:allegato_ffmpeg}

Lorem ipsum dolor sit amet.

\chapter{Acquisizione audio PCM}
\label{cha:allegato_pcm}

Il blocco di codice che segue configura e avvia l'acquisizione della sorgente audio di default del dispositivo, impostando come formato PCM 16bit a $44,1 kHz$ e un canale.

\begin{minted}[xleftmargin=\parindent,linenos]{java}
final int SAMPLING_RATE_IN_HZ = 44100;
final int CHANNEL_CONFIG = AudioFormat.CHANNEL_IN_MONO;
final int AUDIO_FORMAT = AudioFormat.ENCODING_PCM_16BIT;

final int BUFFER_SIZE_FACTOR = 2;
final int BUFFER_SIZE = BUFFER_SIZE_FACTOR *
    AudioRecord.getMinBufferSize(SAMPLING_RATE_IN_HZ, CHANNEL_CONFIG, AUDIO_FORMAT);

AudioRecord recorder = new AudioRecord(
    MediaRecorder.AudioSource.DEFAULT,
    SAMPLING_RATE_IN_HZ,
    CHANNEL_CONFIG,
    AUDIO_FORMAT,
    BUFFER_SIZE);

recorder.startRecording();
isRecording = true;
\end{minted}

L'acquisizione vera e propria dei campioni audio avviene però in un thread separato, in cui vengono caricati i dati audio in un buffer, a ciclo continuo. Un particolare da notare è che il tipo dell'array buffer è \texttt{short[]}, perché deve contenere l'ampiezza del segnale a 16 bit.

Nel momento in cui il buffer deve essere scritto su file (metodo \texttt{writeShortArrayToFile}), viene però effettuata una conversione in byte, assicurandosi di usare l'ordine dei byte "little endian", il più comune nell'ambito dell'audio PCM.

\begin{minted}[xleftmargin=\parindent,linenos]{java}
Thread recordingThread = new Thread(new RecordingRunnable(), "RecordingThread");
recordingThread.start();
\end{minted}

\begin{minted}[xleftmargin=\parindent,linenos]{java}
class RecordingRunnable implements Runnable {
    @Override
    public void run() {
        final File file = new File("/sdcard/audio/test.pcm");

        short[] buffer = new short[BUFFER_SIZE];

        try (final FileOutputStream outStream = new FileOutputStream(file)) {
            while (isRecording) {
                int readSize = recorder.read(buffer, 0, buffer.length);

                if (readSize < 0) {
                    throw new RuntimeException("Reading of audio buffer failed");
                }

                writeShortArrayToFile(buffer, outStream, readSize);
            }

            outStream.close();
        }
    }
    
    private void writeShortArrayToFile(short[] buffer,
                                       FileOutputStream outStream,
                                       int readSize) {
        try {
            ByteBuffer bb = ByteBuffer.allocate(Short.SIZE / Byte.SIZE * readSize);
            bb.order(ByteOrder.LITTLE_ENDIAN);
            ShortBuffer ss = bb.asShortBuffer();
            ss.put(buffer, 0, readSize);
            outStream.write(bb.array(), 0, bb.limit());
        } catch (IOException e) {
            throw new RuntimeException(e);
        }
    }
}
\end{minted}



\end{document}
