\chapter{La libreria \texttt{MobileFFmpeg}}
\label{cha:allegato_ffmpeg}

Lorem ipsum dolor sit amet.

\chapter{Acquisizione audio PCM}
\label{cha:allegato_pcm}

Il blocco di codice che segue configura e avvia l'acquisizione della sorgente audio di default del dispositivo, impostando come formato PCM 16bit a $44,1 kHz$ e un canale.

\begin{minted}{java}
final int SAMPLING_RATE_IN_HZ = 44100;
final int CHANNEL_CONFIG = AudioFormat.CHANNEL_IN_MONO;
final int AUDIO_FORMAT = AudioFormat.ENCODING_PCM_16BIT;

final int BUFFER_SIZE_FACTOR = 2;
final int BUFFER_SIZE = BUFFER_SIZE_FACTOR *
    AudioRecord.getMinBufferSize(SAMPLING_RATE_IN_HZ, CHANNEL_CONFIG, AUDIO_FORMAT);

AudioRecord recorder = new AudioRecord(
    MediaRecorder.AudioSource.DEFAULT,
    SAMPLING_RATE_IN_HZ,
    CHANNEL_CONFIG,
    AUDIO_FORMAT,
    BUFFER_SIZE);

recorder.startRecording();
\end{minted}

L'acquisizione vera e propria dei campioni audio avviene però in un thread separato.

\begin{minted}{java}
Thread recordingThread = new Thread(new RecordingRunnable(), "RecordingThread");
recordingThread.start();
\end{minted}

\begin{minted}{java}
class RecordingRunnable implements Runnable {
@Override
public void run() {
final File file = new File(Environment.getExternalStorageDirectory(), "/audio/" + System.currentTimeMillis() + ".pcm");

short[] buffer = new short[BUFFER_SIZE];

try (final FileOutputStream outStream = new FileOutputStream(file)) {
while (recordingInProgress.get()) {
int readSize = recorder.read(buffer, 0, buffer.length);

if (readSize < 0) {
throw new RuntimeException("Reading of audio buffer failed: " + getBufferReadFailureReason(readSize));
}

writeShortArrayToFile(buffer, outStream, readSize);
}

outStream.close();
}
}
\end{minted}

