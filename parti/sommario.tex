\chapter*{Sommario} % senza numerazione
\label{sommario}

\addcontentsline{toc}{chapter}{Sommario} % da aggiungere comunque all'indice

L'oggetto di questa tesi è lo sviluppo di un sistema di cattura e registrazione multimediale di lezioni universitarie, conferenze e seminari. Il sistema prodotto si basa su un dispositivo fisico dotato di apposito hardware per la cattura video, sul quale viene eseguito un apposito software su sistema operativo Android.

Il sistema è stato realizzato nell'ambito del progetto LODE (Lectures On DEmand) del professor Marco Ronchetti (Università di Trento), ed è stato elaborato durante un tirocinio formativo presso l'azienda AXIA Studio, con l'obiettivo finale di affinare la soluzione e di immetterla sul mercato delle soluzioni IoT accademiche.

Il progetto LODE, nel cui contesto questo lavoro si inserisce, è un progetto con l'obiettivo di fornire una soluzione low cost per l'acquisizione video delle lezioni universitarie. L'ultima versione del sistema prevede la possibilità di registrare flussi multimediali multipli, in particolare il video in uscita dal computer del docente, il video acquisito da una videocamera IP che riprende il docente e/o la lavagna, e l'audio di un microfono indossato dal docente. I contenuti sono poi elaborati e pubblicati su una pagina web accessibile dagli studenti del corso.

Una funzionalità del sistema prevede anche che durante le lezioni lo studente possa catturare degli screenshot di ciò che viene proiettato in quel momento, e inserire delle annotazioni testuali o disegnare sulle catture.

Nella sua ultima iterazione, LODE prevede l'uso di un dispositivo fisico, soprannominato LodeBox, che incorpora un "mini-computer" Raspberry Pi. Tramite apposite estensioni hardware, in particolare un modulo "HDMI to CSI-2", il dispositivo è in grado di acquisire un input HDMI come se si trattasse del video di una videocamera, e di sfruttare funzionalità come l'anteprima su schermo e la registrazione H.264 tramite encoder hardware. Il dispositivo prevede anche la possibilità di collegare un proiettore tramite uscita HDMI, un microfono tramite dongle USB, e una videocamera IP/RTSP (Real Time Streaming Protocol) tramite la rete locale.

Questa soluzione si scontra però con delle difficoltà che rendono difficile la realizzazione di un sistema affidabile e distribuibile su larga scala. Tra le problematiche principali si evidenziano la difficoltà nel gestire situazioni "plug and play", come la disconnessione e la riconnessione del cavo HDMI, la mancanza di storage integrato duraturo e ad alte prestazioni, le scarse garanzie sulla disponibilità del modulo hardware di conversione HDMI.

Ci si è orientati quindi verso la ricerca di SBC (Single-Board Computer) alternative più adatte per la realizzazione di applicazioni multimediali. Tra le soluzioni più accessibili dal punto di vista economico spiccano molte board basate sul sistema operativo Android, che offre i benefici di avere una piattaforma ben nota, documentata e con la possibilità di sostituire l'hardware riusando gran parte del codice scritto.

Gran parte del mio lavoro si è quindi concentrato sullo sviluppo di alcune applicazioni Java/Kotlin, con lo scopo di verificare la fattibilità di una versione di LodeBox con piattaforma Android.

La peculiarità fondamentale di molte board Android con input HDMI è che permettono di sfruttare direttamente le API di Android per l'acquisizione di video e immagini, permettendo di scrivere codice che non sia strettamente legato all'hardware. La piattaforma Android prevede due versioni delle API per l'accesso alla fotocamera, in particolare la classe \texttt{android.hardware.Camera} e le classi contenute in \texttt{android.hardware.camera2.*}. A partire da Android 5.0 (livello API 21), la classe \texttt{Camera} è deprecata ed è invece consigliato l'uso delle API \texttt{camera2}, più avanzate ma anche più complesse da usare. Ciò non ostante, non è garantito che ogni dispositivo con Android 5 o superiore supporti a pieno le API \texttt{camera2}.

È stato quindi necessario approfondire il funzionamento di entrambe le versioni delle API, per determinare se entrambe permettono di soddisfare i requisiti di LODE e quale versione conviene usare a seconda dell'hardware che si ha a disposizione.

Oltre all'input HDMI, si ha la necessità di registrare anche una videocamera esterna, che per contenere i costi è una videocamera IP raggiungibile nella rete locale tramite il protocollo di streaming RTSP. Di conseguenza, il flusso può essere registrato direttamente sulla board, utilizzando una versione della libreria \texttt{ffmpeg} compilata per funzionare su Android.

Un altro punto importante dello sviluppo di un'applicazione embedded per Android è la realizzazione di una modalità \textit{kiosk}, cioè una situazione in cui l'utilizzatore del sistema può utilizzare solo ed esclusivamente una singola applicazione, senza poter accedere alle altre parti del sistema operativo o ad altre funzioni. Questo ha portato a sperimentare diverse soluzioni per assicurarsi che l'applicazione resti sempre a schermo intero, tra cui la modalità "lock task" di Android e l'avvio automatico come applicazione launcher.

Il risultato di questa fase è stato un insieme di applicazioni-prototipo che sono state sperimentate sul campo durante alcune lezioni presso l'Università di Trento, permettendo di raccogliere importanti riscontri sul funzionamento del sistema.

A questo punto la fattibilità delle funzionalità di base che il sistema deve fornire è stata verificata, e i passi successivi hanno riguardato altri aspetti per migliorare il funzionamento del software.

Uno di questi è la possibilità di mettere in pausa e riprendere la registrazione della lezione. L'idea considerata è stata quella di non sospendere la registrazione ma di escludere in post-produzione i segmenti di video corrispondenti alle pause. Questo è agevolmente ottenibile grazie a \texttt{ffmpeg} e con l'ausilio di uno script che generi il comando (potenzialmente lungo) per ritagliare i segmenti. Nel caso in cui non fosse possibile avere una registrazione unica, una tecnica simile può essere adottata per unire i segmenti video in fase di post-elaborazione.

Un'altra questione approfondita riguarda la sincronizzazione dei flussi, con lo scopo principale di evitare che l'audio risulti troppo sfasato rispetto al video, ed evitare che sia visibile lo sfasamento tra voce registrata e labiale del relatore. Il problema sorge principalmente per via della presenza del video RTSP, la cui latenza è difficilmente stimabile. La soluzione sperimentale proposta prevede di incorporare l'audio nella registrazione video HDMI (in modo da ottenere una naturale sincronizzazione tra i due flussi acquisiti via cavo), e di occuparsi invece di sincronizzare video HDMI e RTSP. L'implementazione sperimentale è stata ottenuta mostrando su schermo un marcatore visivo che permetta di individuare con precisione un instante comune tra i due flussi video, e di conseguenza sincronizzarli.

Come accennato, un'altra funzione del sistema LODE prevede che lo studente possa prendere appunti in tempo reale durante lo svolgimento delle lezioni, catturando degli screenshot di ciò che vede in quel momento. Per evitare di inviare inutilmente screenshot al server nel caso in cui non ci sia stato nessun cambiamento percepibile nell'immagine, un sistema intelligente può rilevare le differenze tra i fotogrammi per determinare se l'immagine è nuova o invariata. Dopo aver appurato empiricamente che il confronto di tutti i pixel di due fotogrammi risulta computazionalmente dispendioso, un'alternativa considerata è quella di scegliere con un fattore di casualità un numero limitato di pixel "salienti". A livello intuitivo, il metodo sviluppato realizza una griglia con un numero oscillante di righe e colonne, le cui intersezioni determinano una quantità limitata di pixel che possono essere utilizzati per rilevare rapidamente le differenze tra i fotogrammi.

I capitoli di questa tesi approfondiscono più dettagliatamente le problematiche e le soluzioni che sono state sintetizzate in questo sommario.

%\begin{itemize}
%  \item scelta delle API di Android Camera o Camera2, per l'accesso all'input HDMI del dispositivo;
%  \item implementazione della registrazione di un flusso video RTSP, per poter acquisire una videocamera connessa in rete;
%  \item implementazione della registrazione del video HDMI in ingresso;
%  \item aggiunta della cattura di fotogrammi durante la registrazione della lezione;
%  \item	integrazione dell'applicazione con il web server che permette agli studenti di catturare screenshot;
%  \item	rilevamento della disconnessione e riconnessione dell'input HDMI, e del cambio risoluzione;
%  \item	realizzazione di una “kiosk mode”, in modo da impedire all'utilizzatore di uscire dall'applicazione o di compromettere il sistema o il dispositivo;
%  \item	realizzazione di uno script per ritagliare segmenti dei video registrati, per ottenere la funzionalità di pausa/ripresa della registrazione;
%  \item	sincronizzazione dei flussi video HDMI e RTSP, in particolare mediante l'uso di un marcatore visivo presente in entrambi i flussi;
%  \item	rilevamento ed estrazione delle “slide” dal video HDMI in post-produzione;
%  \item	rilevamento dei cambiamenti al flusso video HDMI, in modo da evitare di inviare al server screenshot che sono già stati inviati;
%  \item	registrazione di audio LPCM e compressione AAC.
%\end{itemize}

